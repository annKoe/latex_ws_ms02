\section{"`Migration in 7 Schritten"'}

% ==================================================
\subsection{Probleme/ Herausforderungen}

\begin{itemize}
  \item welche Service deployed und wo laufen diese
  \item \ldots
  \item Konfiguration
\end{itemize}

% ==================================================
\subsection{Challenge Nr. 1: "`Integration"'}
\subsubsection{Integration a.k.a Kommunikation}
mehrere Services - m�ssen miteinander kommunizieren

% --------------------------------------------------

\subsubsection{Kommunikation via REST}
restlet.com - HttpStatusCodes
synchrone Kommunikation

% --------------------------------------------------

\subsubsection{Kommunikation via Messaging}
\begin{itemize}
  \item lose Kopplung durch Queues (Message kann nur von einem genommen werden
  - ist dann raus) und Topics (Message kann von mehreren Consumern genommen
  werden)
  \item asynchrone Kommunikation
  \item garantierte Auslieferung m�glich
\end{itemize}

% --------------------------------------------------

\subsubsection{UI Integration via Client/ Browser}

\begin{itemize}
  \item Hyperlinks (je eine Seite je Service)
  \item Client Side Composition (Angular, Node)
  \item iFrame ("`fake"' Document Component Model)
  \item AJAX ("`blank"' Page)
  \item JavaScript Voodoo (a.k.a "`SvenJa"')
  \item HTML-Imports (Teil der WebComponents Spec) - Template-Mechanismus
\end{itemize}

% --------------------------------------------------

\subsubsection{UI Integration via Server Side Include}
Apache httpd, LiteSpeed, nginx, lighttpd, IIS

\subsubsection{UI Integration via Edge Side Include}
Proxy Caching Server (Varnish, Squid, Mongrel) - propriet�re L�sungen

% --------------------------------------------------

\subsubsection{UI Integration via UI/ API Gateway}
Vorsicht:
\begin{itemize}
  \item Web UI/ API Monolith
  \item Kein Platz f�r Logik
\end{itemize}

Single Point of Entry

% --------------------------------------------------

\subsubsection{Data Integration via gemeinsame Datenquelle}

\begin{itemize}
  \item keine unabh�ngige Entwicklung
  \item nicht losgel�st
  \item �nderungen betreffen mehrere Services
\end{itemize}

% --------------------------------------------------

\subsubsection{Data Integration via getrennte Datenquelle}
\begin{itemize}
  \item lesend und schreibend via DataOwnerService
  \item = sehr viel Kommunikation
\end{itemize}

% --------------------------------------------------

\subsubsection{Data Integration via Replikation}
\begin{itemize}
  \item lesender Zugriff - auf eigene Datenbank
  \item schreibender Zugriff via Messaging (DataOwnerService)
  \item verteilte "`Transaktionen"' via Events
  \item eventuell Konsistenze Daten
\end{itemize}

% ==================================================

\subsection{Challenge Nr. 2: "`Discovery"'}

\subsubsection{Herausforderungen}
Service-Adressen werden dynamisch vergeben = Service Registry

\paragraph{ClientSideDiscovery}
Vorteil - Client kann f�r sich selber entscheiden, welcher Service verwendet
wird

\paragraph{ServerSideDiscovery}
eine Art "`Router"' wird dazwischen geschalten


\paragraph{LoadBalancing}
\begin{itemize}
  \item selbst implementiert
  \item Tools nutzen (Ribbon, Feign)
\end{itemize}

% ==================================================

\subsection{Challenge Nr. 3: "`Resilience"'}

\begin{itemize}
  \item Design for failure
  \item fail sooner than later
  \item meaningful response (fallback defaults)
\end{itemize}

\begin{itemize}
  \item Load Balancing  
  \item Retry on Failure (temporary)
  \item Timeouts
  \item Circuit Breaker
  \item Bulk Heads - getrennte Threadpools (Hystrix)
\end{itemize}

% ==================================================

\subsection{Challenge Nr. 4: "`Security"'}



% ==================================================
\subsection{Challenge Nr. 5: "`Monitoring"'}

h�ngt eng mit Logging zusammen - muss zusammengef�hrt werden
gemeinsamer Aufh�nger wird ben�tigt - Correlation IDs



% ==================================================

\subsection{Tools}
\subsubsection{Tools}

\paragraph{Swarm}

\paragraph{Postmann} 
- Visualisierung der GET und POST-Aufrufe Ausgabe von Status-Codes, Headers usw.

\paragraph{Consul} - Service Registry

\paragraph{Ribbon} 

\paragraph{Feign}

\paragraph{MicroProfileIO}

\paragraph{KEYCLOAK}
